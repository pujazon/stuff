 \documentclass[12pt]{article}
\usepackage[english]{babel}
\usepackage[utf8x]{inputenc}
\usepackage{amsmath}
\usepackage{graphicx}
\usepackage{float}
\usepackage[colorinlistoftodos]{todonotes}

\begin{document}
\begin{titlepage}
\newcommand{\HRule}{\rule{\linewidth}{0.5mm}} 

\center 

\textsc{\LARGE Universistat Polit\`{e}cnica de Calatunya}\\[1.5cm]
\includegraphics[scale=0.2]{Upc.PNG} \\[1cm]
\textsc{\Large Statistical Modelling and Design of Experiments}\\[0.5cm] 
\textsc{\normalsize Master in Innovation and Research in Informatics}\\[0.5cm] 

\HRule \\[0.4cm]
{ \Large \bfseries Assignment 2}\\[0.4cm]
\HRule \\[1.5cm]

\begin{minipage}{0.4\textwidth}
\begin{flushleft} \large
\emph{Author:}\\
First name \textsc{Daniel Pujazon Bonet}\\
\end{flushleft}

\end{minipage}\\[2cm]

{\large \today}\\[2cm]
\vfill

\end{titlepage}

\section {Generate your data.}
\vspace{5mm}
\subsection {Define, for each factor (from 1 to 5) a distribution (the RVGs that you prefer, uniform, normal, exponential, etc.). For the factors 6 to 10 define a function that uses the previous variables, as an example F6=F1+2F3.}
\subsection {Define an answer variable that will be composed by a function that combines a subset of the previous factors plus a normal distribution you know (to add some random noise).}
The data stream is generated through the attached DataGen.cpp file and it's save it on a .csv file called DataSet.csv (also there's a .txt file).
(Comment each factor which distribution has followed).

\section {Obtain an expression to generate new data.}
\vspace{5mm}
\textbf{Imagine that you don’t know nothing regarding how this dataset has been generated. Consider that the factors represent different machines and the answer is the time to do an operation.\\
You need to explore it because you want to define a model to obtain new data for your DOE (you want to detect the possible relations and the interactions between the factors, or maybe you want to test alternatives or predict future scenarios).}
\subsection {Explore the possible relations of all the factors and the answer variable, you can use any technique developed during the course (LRM or ANOVA).}

We are gonna to do a PCA due to know the iterrelation between the different factors. After that we will generate some Linear Regresion models (LRM) and we will take the one that fits better our requirements. All will be made through the R script ObtainExpresion.r\\
So, applying PCA to the DataSet.csv we have found:
\begin{itemize}
\item F3, F8, F6, F1, F7 and F2 are strong correlated. F3,F8 and F6 are positive correlated while F1, F7 and F2 are negative correlated.\\ On the other side, F0, F5, F4 and F9 are strong and positive correlated.
\item Each group angle respect to one of the axis is near to 0, what means that these group is fully explained only by one PC: First group (F3,F8...) is almost parallel to y-axis, PC2, while second one (F0, F5...) is almost parallel to x-axis, PC1. 
This means that the other-PC group variables has no effect (because they're independant or because the impact of the other variable on the result it's too weak) on the value of the ohter-PC group vairables (pe: F5 value is not significantly determined by F3, F6 is not determined by F0... and so on)
\item Scree plot show us that we need at least 4 variables to explain more than the 80\% of the system variables variation (even if we go further, with 5 variables we explain 100\% of variability so this means that from 6th variable, which explain 0\% of the variance, are full explained by the other variables).
\end{itemize}
So, knowing that there are 5 variables that explain 100\%, we could do all the possible LRM using 5 of the 10 variables (which would be 252, combination without repetitions of 10 elements taken from 5 to 5) and we will get the one that fit better. \\

A LRM with all variables of course exlpain 100\% of the variation on answer.
A LRM2 with all the first 5 variables explain also the 100\%, so these was the combination desired from the 252 potential combinations.
On LRM3 if we only take the first 4 factors (as scree plot conclusions). With that we only can explain the 78\%, so we will use the LRM2:

Answer = 5.00e+00(X1)-3.00e+00(X2)-5.00e+00(X3)-1.18e-06(X4)+1.00e+00(X5)

\subsection {Describe what you find on this analysis and, explain if it is coherent with the knowledge you have from the data.}
(Confirm it, both PCA and LRM doing "test", input the data of 5 factors and get the output).

\subsection {Use a simulation model to generate new data. The simulation model will be a very simple model composed by one server by each one of the factors you use on the answer.}
(First let's specify that we are gonna to do a $v*2^{k}$ factory design. Then define the maximum and minimum value for each factor of the LRM. Finally do the GPSS and explain how is performed.)

\end{document}